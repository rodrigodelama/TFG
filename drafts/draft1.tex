%----------
%   WARNING
%----------

% This Guide contains Library recommendations based mainly on APA and IEEE styles, but you must always follow the guidelines of your TFG Tutor and the TFG regulations for your degree.

% THIS TEMPLATE IS BASED ON THE IEEE STYLE 

% TFG - Machine Learning Based Predictive Modeling of Energy Prices

%----------
% DOCUMENT SETTINGS
%----------

\documentclass[12pt]{report} % font: 12pt

% margins: 2.5 cm top and bottom; 3 cm left and right
\usepackage[
a4paper,
vmargin=2.5cm,
hmargin=3cm
]{geometry}

% Paragraph Spacing and Line Spacing: Narrow (6 pt / 1.15 spacing) or Moderate (6 pt / 1.5 spacing)
\renewcommand{\baselinestretch}{1.15}
\parskip=6pt

% % New Paragraph spacing
% % \usepackage{parskip}{1em plus 0.2em minus 0.1em}
% \setlength{\parskip}{1em}
% \setlength{\parindent}{20pt}

% Color settings for cover and code listings 
\usepackage[table]{xcolor}
\definecolor{azulUC3M}{RGB}{0,0,102}
\definecolor{gray97}{gray}{.97}
\definecolor{gray75}{gray}{.75}
\definecolor{gray45}{gray}{.45}

% PDF/A -- Important for its inclusion in e-Archive. PDF/A is the optimal format for preservation and for the generation of metadata: http://uc3m.libguides.com/ld.php?content_id=31389625.

% In the template we include the file OUTPUT.XMPDATA. You can download that file and include the metadata that will be incorporated into the PDF file when you compile the memoria.tex file. Then upload it back to your project.
\usepackage[a-1b]{pdfx}

% LINKS
\usepackage{hyperref}
\hypersetup{colorlinks=true,
	linkcolor=black, % links to parts of the document (e.g. index) in black
	urlcolor=blue} % links to resources outside the document in blue

% MATH EXPRESSIONS
\usepackage{amsmath,amssymb,amsfonts,amsthm}

% Character encoding
\usepackage{txfonts} 
\usepackage[T1]{fontenc}
\usepackage[utf8]{inputenc}

% English settings
\usepackage[english]{babel} 
\usepackage[babel, english=american]{csquotes}
\AtBeginEnvironment{quote}{\small}

% Footer settings
\usepackage{fancyhdr}
\pagestyle{fancy}
\fancyhf{}
\renewcommand{\headrulewidth}{0pt}
\rfoot{\thepage}
\fancypagestyle{plain}{\pagestyle{fancy}}

% DESIGN OF THE TITLES of the parts of the work (chapters and epigraphs or sub-chapters)
\usepackage{titlesec}
\usepackage{titletoc}
\titleformat{\chapter}[block]
{\large\bfseries\filcenter}
{\thechapter.}
{5pt}
{\MakeUppercase}
{}
\titlespacing{\chapter}{0pt}{0pt}{*3}
\titlecontents{chapter}
[0pt]                                               
{}
{\contentsmargin{0pt}\thecontentslabel.\enspace\uppercase}
{\contentsmargin{0pt}\uppercase}                        
{\titlerule*[.7pc]{.}\contentspage}                 

\titleformat{\section}
{\bfseries}
{\thesection.}
{5pt}
{}
\titlecontents{section}
[5pt]                                               
{}
{\contentsmargin{0pt}\thecontentslabel.\enspace}
{\contentsmargin{0pt}}
{\titlerule*[.7pc]{.}\contentspage}

\titleformat{\subsection}
{\normalsize\bfseries}
{\thesubsection.}
{5pt}
{}
\titlecontents{subsection}
[10pt]                                               
{}
{\contentsmargin{0pt}                          
	\thecontentslabel.\enspace}
{\contentsmargin{0pt}}                        
{\titlerule*[.7pc]{.}\contentspage}  


% Tables and figures settings
\usepackage{multirow} % combine cells 
\usepackage{caption} % customize the title of tables and figures
\usepackage{floatrow} % we use this package and its \ ttabbox and \ ffigbox macros to align the table and figure names according to the defined style.
\usepackage{array} % with this package we can define in the following line a new type of column for tables: custom width and centered content
\newcolumntype{P}[1]{>{\centering\arraybackslash}p{#1}}
\DeclareCaptionFormat{upper}{#1#2\uppercase{#3}\par}
\usepackage{graphicx}
\graphicspath{{imagenes/}} % Images folder

% Table layout for engineering
\captionsetup*[table]{
	format=upper,
	name=TABLE,
	justification=centering,
	labelsep=period,
	width=.75\linewidth,
	labelfont=small,
	font=small
}

% Figures layout for engineering
\captionsetup[figure]{
	format=hang,
	name=Fig.,
	singlelinecheck=off,
	labelsep=period,
	labelfont=small,
	font=small		
}

% FOOTNOTES
\usepackage{chngcntr} % continuous numbering of footnotes
\counterwithout{footnote}{chapter}

% CODE LISTINGS 
% support and styling for listings. More information in  https://es.wikibooks.org/wiki/Manual_de_LaTeX/Listados_de_código/Listados_con_listings
\usepackage{listings}

% Custom listing
\lstdefinestyle{estilo}{ frame=Ltb,
	framerule=0pt,
	aboveskip=0.5cm,
	framextopmargin=3pt,
	framexbottommargin=3pt,
	framexleftmargin=0.4cm,
	framesep=0pt,
	rulesep=.4pt,
	backgroundcolor=\color{gray97},
	rulesepcolor=\color{black},
	%
	basicstyle=\ttfamily\footnotesize,
	keywordstyle=\bfseries,
	stringstyle=\ttfamily,
	showstringspaces = false,
	commentstyle=\color{gray45},     
	%
	numbers=left,
	numbersep=15pt,
	numberstyle=\tiny,
	numberfirstline = false,
	breaklines=true,
	xleftmargin=\parindent
}

\captionsetup*[lstlisting]{font=small, labelsep=period}

\lstset{style=estilo}
\renewcommand{\lstlistingname}{\uppercase{Código}}


% REFERENCES 

% IEEE bibliography setup
\usepackage[backend=biber, style=ieee, isbn=false,sortcites, maxbibnames=6, minbibnames=1]{biblatex} % Setting for IEEE citation style, recommended for engineering. "maxbibnames" indicates that from 6 authors truncate the list in the first one (minbibnames) and add "et al." as used in the IEEE style.

\addbibresource{referencias.bib} % The references.bib file in which the bibliography used should be


%----------
%	DOCUMENT
%----------

\begin{document}
\pagenumbering{roman} % Roman numerals are used in the numbering of the pages preceding the body of the work.
	
%----------
%	COVER
%----------	
\begin{titlepage}
	\begin{sffamily}
	\color{azulUC3M}
	\begin{center}
		\begin{figure}[H] % UC3M Logo
			\makebox[\textwidth][c]{\includegraphics[width=16cm]{logo_UC3M.png}}
		\end{figure}
		\vspace{2.5cm}
		\begin{Large}
			University Degree in Telematics Engineering\\			
			 2024-2025\\ % Academic year
			\vspace{2cm}		
			\textsl{Bachelor Thesis}
			\bigskip
			
		\end{Large}
		 	{\Huge ``Machine Learning-Based Predictive Modeling of Energy Prices''}\\
		 	\vspace*{0.5cm}
	 		\rule{10.5cm}{0.1mm}\\
			\vspace*{0.9cm}
			{\LARGE Rodrigo De Lama Fernández}\\ 
			\vspace*{1cm}
		\begin{Large}
			Emilio Parrado Hernández\\
			Madrid, España, May 31st 2025\\
		\end{Large}
	\end{center}
	\vfill
	\color{black}
	% IF OUR WORK IS TO BE PUBLISHED UNDER A CREATIVE COMMONS LICENSE, INCLUDE THESE LINES. IS THE RECOMMENDED OPTION.
	\includegraphics[width=4.2cm]{creativecommons.png}\\ % Creative Commons Logo
    This work is licensed under Creative Commons \textbf{Attribution – Non Commercial – Non Derivatives}
	\end{sffamily}
\end{titlepage}

\newpage % blank page
\thispagestyle{empty}
\mbox{}

%----------
%	ABSTRACT AND KEYWORDS 
%----------
% summary es algo mas extendido - miro lo que diga la uni
\renewcommand\abstractname{\large\bfseries\filcenter\uppercase{Summary}}
\begin{abstract}
\thispagestyle{plain}
\setcounter{page}{3}
	
	% Write your abstract
    This project focuses on the development of a machine learning-based predictive model for electricity prices in Spain. Using historical data from the OMIE and technical analysis (TA) indicators, the model aims to accurately forecast hourly energy prices. The study focuses on a single hourly slot, evaluating the performance of various Machine Learning models, such as linear regression, Lasso, and Random Forest. There was a strong focus on the feature engineering, employing technical analysis indicators such as moving averages, exponential moving averages, and momentum metrics. The results display the impact of tailoring the features to improve model accuracy and offer insights into the potential of data-driven approaches for energy price forecasting.
	
	\textbf{Keywords:} % add the keywords
            
            Energy
            
            Machine Learning
            
            Predictive Modeling

            Technical Analysis (TA)
            
            % NOT RELEVANT
            % o	Forecasting
            % o	Neural networks
	
	\vfill
\end{abstract}
\newpage % Blank page
\thispagestyle{empty}
\mbox{}


%----------
%	Dedication
%----------	
\chapter*{Dedication}

\setcounter{page}{5}

% Write here
\noindent Dedicated to my late grandfather who was not able to see me become an engineer like himself\\
Here is to my amazing family\\
To my parents, that have always helped me push through hard moments
To my grandparents\\
To my girlfriend\\
To my friends and colleagues\\
To anyone and everyone that has supported me during my years in university\\

\noindent Here is to the next steps in life

	\vfill

\newpage % blank page
\thispagestyle{empty}
\mbox{}


%----------
%	TOC
%----------	

\tableofcontents
\thispagestyle{fancy}

%----------
% Estructura TFG (basada en tesis de mama)
%----------

    % Agradecimientos
    
    % Dedicatorias
    
    % Indice
    % -	Abstract of the research
    % -	Key words
    
    % 1.	Introduction
    % o	Why am I doing this 
    % 	Personal
    % 	Professional
    % 2.	Marco teorico
    % a.	Teoría de los modelos preditivos
    % b.	Otros estudios
    % c.	Que tipos de modelos hay (y explicar la teoría de los distintos tipos, buscando otros autores que hayan hecho estudios con dichas metodologías y citándolos)
    % i.	Technical Analysis to create a Machine Learning model (24 in parallel)
    % ii.	Time Series Forecasting
    % iii.	Neural Networks
    % 3.	MI Modelo para predecir el precio de la electricidad
    % a.	Metodologia
    % b.	Objeto de estudio
    % c.	Plan de investigación (not really pero valorar)
    % i.	objetivos
    % ii.	preguntas
    % iii.	metodologia
    % 4.	Analysis and Interpretation of the obtained results
    % 5.	Conclusiones de la investigacion y del modelo
    %   a.	Areas de investigación alternativas que se puedan investigar
    % 6.	Referencias Bibliograficas
    % 7.	Glosario de términos
    %   a.	Explicar cosas en detalle
    % 8.	Referencias

%----------
% Estructura de Emilio
%----------

    % Acknowledgements
    
    % Dedication (to… mom dad gf grandparents my teacher etc)
    
    % Index
    % 1.	Introduction
    %     a.	Context (del problema)
    %     b.	Motivation (de la solucion)
    %     c.	Objectives
    %     d.	Summary of the results
    % 2.	Background
    %     a.	Description of the different componentes que uso a modo sencillo - eg modelos etc
    %     b.	ML
    %     c.	Energia
    % 3.	Proposal - si me cambian los datos este capitulo no deberia cambiar
    %     a.	Theoretical system description - esto es como juntas estos componentes - la matriz, el feature selection, los modelos especificos
    %     b. si es una limpieza generica en el tres
    % 4.	Experimentation
    %     a.	Data Description - descripcion de datos y parametros de los resultados - si es una limpieza especifica en el 4
    %     b.	Set Up Explanation - para el RF que rangos de parametros voy a hacer, dias y profundidad, validacion cruzada etc
    %     c.	Results 
    %     d.	Low level discussion
    % 5.	Conclusions
    %     a.	Recap
    %     b.	Revisit Objectives
    %     c.	Future Work
    %           - mejorar el error
%----------



% BLANK PAGE
\newpage % blank page
\thispagestyle{empty}
\mbox{}

%----------
% List of figures. If they are not included, comment the following lines
%----------
\listoffigures
\thispagestyle{fancy}

%     Example of figure:
%     \begin{figure}[H]
%     	\ffigbox[\FBwidth] {
%     	\caption[Name as seen in Index]{Figure name}
%     	}
%     	{\includegraphics[scale=0.6]{imagenes/creativecommons.png}}
%     \end{figure}

%----------
% List of tables. If they are not included, comment the following lines
%----------
\listoftables
\thispagestyle{fancy}

%     Example of table:
% \begin{table}[H]
% 	\ttabbox[\FBwidth]
% 	{\caption{Lorem ipsum}}
% 	{\begin{tabular}{|c|P{1.5cm}|c|P{1.5cm}|P{2cm}|c|P{1.5cm}|P{2cm}|}
% 		\hline
% 		\multicolumn{2}{|c|}{\textbf{I}} & \multicolumn{2}{c|}{\textbf{II}} & \multicolumn{3}{c|}{\textbf{III}} & \textbf{IV} \\
% 		\hline
% 		x & y & x & y & x & y & x & y \\
% 		\hline
% 		10.0 & 8.04 & 10.0 & 9.14 & 10.0 & 7.46 & 8.0 & 6.58 \\
% 		\hline
% 		8.0 & 6.95 & 8.0 & 8.14 & 8.0 & 6.77 & 8.0 & 5.76 \\
% 		\hline
% 		13.0 & 7.58 & 13.0 & 8.74 & 13.0 & 12.74 & 8.0 & 7.71 \\
% 		\hline
% 		9.0 & 8.81 & 9.0 & 8.77 & 9.0 & 7.11 & 8.0 & 8.84 \\
% 		\hline
% 		11.0 & 8.33 & 11.0 & 9.26 & 11.0 & 7.81 & 8.0 & 8.47 \\
% 		\hline
% 		14.0 & 9.96 & 14.0 & 8.10 & 14.0 & 8.84 & 8.0 & 7.04 \\
% 		\hline
% 		6.0 & 7.24 & 6.0 & 6.13 & 6.0 & 6.08 & 8.0 & 5.25 \\
% 		\hline
% 		4.0 & 4.26 & 4.0 & 3.10 & 4.0 & 5.39 & 19.0 & 12.50 \\
% 		\hline
% 		12.0 & 10.84 & 12.0 & 9.13 & 12.0 & 8.15 & 8.0 & 5.56 \\
% 		\hline
% 		7.0 & 4.82 & 7.0 & 7.26 & 7.0 & 6.42 & 8.0 & 7.91 \\
% 		\hline
% 		5.0 & 5.68 & 5.0 & 4.74 & 5.0 & 5.73 & 8.0 & 6.89 \\
% 		\hline
% 		\multicolumn{5}{l}{Source: BOE}
% 	\end{tabular}}
% \end{table}

%----------
%	THESIS
%----------	
\clearpage
\pagenumbering{arabic} % numbering with Arabic numerals for the rest of the document.	

% IMPORTANT: Latex special characters are: # $ % & \ ^ _ { } ~. To avoid mistakes when compiling try writing \ before. For: \ use \textbackslash ; for ^ \textasciitilde and ~ \textasciicircum.

% Start writing here----------------------------------------------------



% Chapter 1 - introduction to the problem
\chapter{Introduction to the problem}
    \begin{itemize}
    \item \textbf{1. Explanation of how the price of energy works in Spain}
        \begin{itemize}
        \item a. Energy
        \item b. Prices
        \end{itemize}
    \item \textbf{2. Machine Learning - Lets run through some model overviews, which are available to choose, but without actually referencing anything}
    \end{itemize}

% We thought about different approaches such as Time-Series Forecasting, about Neural Networks but ...
Energy has increasingly become more important.
Availability is key for a modern high functioning society.q
In Europe we are lucky to have a really advanced and reliable network.
It is a key resource needed to succeed in the ecological transition that we are currently undergoing world wide.
Electric mobility is a key element of this process.
Electric cars, buses and trucks, need easily available and convenient sources of energy to recharge, to be able to play their role in society.
An important factor in this ever growing necessity is the price of the available energy.
But in recent times, that easy and affordable access has changed, and has become more unpredictable.
Ever since the commencement of the war between Russia and Ukraine marked and inflection point in the mostly cyclical nature of electricity prices.
Prices have gone up considerable since a couple years time, and it has been affecting everyone, from big consumers such as companies, to everyday people, having to pay a more expensive electricity bill at the end of the month.
This has become a topic of great concern, with more and more people thinking about it often.

This led me to think about pricing, how it was structured, and what to expect as an end consumer.
Could we plan our use of energy according to the price?
Could it be predicted accurately?
These were some of the questions I had that sparked my interest in looking into energy predictions.

In the initial discussions of this project, we talked about various ways to attack the problem at hand.
We thought about a variety of systems but settled on a plan to explore the predictions with the more established methods of Machine Learning.
This was done specially since it was a natural continuation of what was learned in the third year subject of Telematics Engineering, Modern Theory of Detection and Estimation.

The first idea we developed was to separate out the project in 24 blocks, one for each hour block to predict.
The reason behind splitting the data in 24 blocks was simple, we assumed that data between the same time slot across neighbor days, would be similar, or would follow a certain trend.

Focusing on the innovation side of things, we decided to simplify the problem.
Pivoting from creating an algorithm to predict hourly prices, to splitting that into the 24 blocks, and predicting a single one.
We decided on the 14th hour of each day, since irrespective of the day of the week, or holiday, it would always be a valley period.

This kind of granular breaking up of data is similar to what a Random Forest does in order to absorb more details and create higher resolution predictions. We made this assumption based on the idea that it would simplify the project substantially.
% This forced us to generalize our approach, creating a system to analyze and evaluate the data in 24 distinct blocks, figuring out the best matrix dimensions for our linear regression matrix.

Talk about Emilio's background in investment banking at BBVA and his alorithms.

Talk about the idea of using technical analysis for feature engineering



% Chapter 2 - Background and Related Work (State of the Art???)
\chapter{Background and Related work}
\begin{itemize}
\item \textbf{References}
    \begin{itemize}
        \item How is the price of electricity / energy set? What happens if new producers enter the market?
        \item Articles that reference energy prices prediction as an inspiration for my own work:
        \begin{itemize}
            \item 5-10 articles that do/talk about similar things
            \item Energy consumption predictions
            \item Energy prices in other European countries (Italy, Germany, UK, France)
            \item Longer term predictions
            \item Predicciones de produccion
        \end{itemize}
        \item Describir tecnologias – explicar mas en detalle como funciona y enlaces a donde pueda averiguar el contenido
    \end{itemize}

    \item Referenciar que metricas uso para medir mi error etc
    \item Referencias de donde saco los datos

    \item Referencias a que software utilizo referencias a documentacion de sklearn etc
\end{itemize}

\section{General background}
How is the price of electricity set in Spain and Portugal?

What happens when new producers enter the market and sell to the energy pool?

Comment about more typical case studies focusing on energy consumption, since that is something more consistent and predictable

%-------

\section{The Sliding Window plan: Setting up the Feature Matrix}

How do I prepare my data to utilize across different models: The Feature Matrix (weight matrix)

The sliding window theory (mini-models)

\section{Which Machine Learning models will be used}
What algorithms will I use?

- Linear Regression

- Lasso Regression

- Random Forest

\section{Fine Tuning with Technical Analysis and Feature engineering}
What Type of finetuning will I use?

- Feature engineering \& selection

- Technical analysis based feature engineering
    
    TA metrics used? SMA, EMA, ROC \& RSI

    Talk about the intervals used:
    
    $ SMA_3,SMA_5,SMA_7,SMA_{14},SMA_{30},SMA_{60},SMA_{90},SMA_{180},SMA_{360} $
    
    $ EMA_3,EMA_5,EMA_7,EMA_{14},EMA_{30} $
    
    $ ROC_3,ROC_5,ROC_7,ROC_{12},ROC_{14},ROC_{30} $
    
    $ RSI_5,RSI_7,RSI_{14} $

%---

\section{Evaluation Metrics}
How will I measure my accuracy? errors:
$ R^2 $
RMSE \& MSE



% Chapter 3 - Proposal - Theoretical System Description ?
\chapter{Theoretical System Proposal Description}

This chapter will outline the theoretical design of the system developed for the project. It will outline the complete pipeline constructed for the system: starting from the data preparation stage, continuing with the data preparation to the final results.

%---

Data Preparation

Matrix Creation

Mini-models - sliding window approach
No estacionario. No son datos IID - Independientes e identicamente distribuidos.
Hay correlacion en mis datos entre cada dia? Si deberia ser relativamente alta, estacional, pero progresiva y lenta.

Technical indicators
Que es cada cosa y porque usarlos en vez de series temporales, porque la info temporal esta implicita en los indicadores.
Indicadores que capturan patrones en la serie
Ventaja: modelo ML mas sencillo pq las features son muy informativas vs

ML blocks
    Linear Regression
        Linear combination of every feature
    Lasso Regression
        Can turn unimportant features into 0s
    Random Forest
        Feature engineering

Pandas, NumPy, scikit-learn

%---

Retrieve Data from OMIE

Ingest data and create a CSV db

Create Sliding Window Matrix for use in training that batch

Train the batch

Cross validation?? Not really done, in another way w the sliding window
cambio parametros pero no la arquitectura del modelo

Review the mini model theory, why is it reasonable to not use training set, validation and test sets?

Review error rates and compare with other iterations/ batches and other models.

(make a block diagram of all this)

\noindent \textbf{Detailed Steps:}
%----------------------------------

Prior set up:

Developed and tested with Python 3.13 \cite{python}

To set up a new virtual environment \cite{python_venv} run:
\begin{verbatim}
python -m venv .venv
\end{verbatim}

To activate the new virtual environment run:
\begin{verbatim}
in macOS
source .venv/bin/activate
in Windows
.venv\Scripts\Activate.ps1
\end{verbatim}

Install the following contents to reproduce the original virtual environment:
\begin{verbatim}
# To install the following packages run:
# pip install -r requirements.txt 
pandas
scikit-learn
numpy
matplotlib
ta
ipykernel
xgboost
lightgbm
requests
\end{verbatim}

1. Download of OMIE data
\cite{omie_datos}

Downloaded with a Python script for the dates that were not available with an easy.zip file (for whole previous years)

Insert part of the python script?? or better in point 4


2. The data has this shape:
\begin{verbatim}
MARGINALPDBC;
2018;01;01;1;28.1;6.74;
2018;01;01;2;33;4.74;
2018;01;01;3;32.9;3.66;
2018;01;01;4;28.1;2.3;
2018;01;01;5;27.6;2.3;
2018;01;01;6;24.6;2.06;
2018;01;01;7;20.1;2.06;
2018;01;01;8;19.9;2.06;
2018;01;01;9;19.84;2.3;
2018;01;01;10;19.9;2.3;
2018;01;01;11;19.9;2.3;
2018;01;01;12;19.9;2.3;
2018;01;01;13;23.6;2.3;
2018;01;01;14;25.1;2.3;
2018;01;01;15;23.6;5;
2018;01;01;16;24.6;5;
2018;01;01;17;25.1;5;
2018;01;01;18;27.6;8.85;
2018;01;01;19;27.6;15.93;
2018;01;01;20;28.1;22.02;
2018;01;01;21;32.9;20;
2018;01;01;22;28.1;21.95;
2018;01;01;23;28.1;23.52;
2018;01;01;24;27.6;16.35;
*
\end{verbatim}

3. Data format: In page number 67, chapter 6.18 we may find the following information regarding the encoding format for the information: 
\cite{omie_formatos_2024}
\begin{verbatim}
6.18 Precios marginales del mercado diario (MARGINALPDBC)

Fichero con los precios marginales del Mercado Diario para cada una
de las horas.
Nombre del fichero: marginalpdbc_aaaammdd.v donde aaaammdd
corresponde a la fecha de sesión y v es la versión del fichero.

Descripción de los campos:
CAMPO DESCRIPCIÓN VALORES VÁLIDOS
Año Año I4 – 20XX
Mes Mes I2 – 1 a 12
Día Día I2 – 1 a 31
Hora Hora I2 – 1 a 25
MarginalPT Precio marginal zona Portuguesa F8.2 –
-99999.99 a 99999.99
MarginalES Precio marginal zona Española F8.2 –
-99999.99 a 99999.99
\end{verbatim}

4. Data organization

We will transform the data into a CSV format easier to use with the Pandas library to later create a DataFrame. We will maintain the same information, but in a encoding more suitable for Pandas. For that, the date and time must be transformed to a format such as Datetime which will transform the information from \texttt{2018;01;01;14; ...} to \texttt{2018-01-01 14:00:00}

5. Data visualization with Seaborn? With Microsoft Corporation's "Data Wrangler" VSCode extension

6. Data retention of only the relevant info time and date, and MarginalES - drop MarginalPT

7. Data cleanup:
Check for missing or duplicate entries. since Daytime Savings introduces duplicate hours some days of the year:
INSERT some dates in the dataset that have the duplicated entries

8. Explain all blocks of the system?
    Data download:
    Data ingest

Contar los bloques empleados en cada parte – como junto las piezas para construir el sistema – porque uso las piezas que uso

Hablar de implementaciones como que librerías, que scripts etc – no es muy necesario, mas importante la idea
%----------------------------------



% Chapter 4 - Experimentation
\chapter{Experimentation}
\section{Data Description – OMIE data}
How is the data provided?

\section{Set Up Explanation}
How do I use the previously explained system - what parameters did I modify to obtain the final results. Feature engineering, Alpha variation testing in Lasso, Tree depth and number of leafs.
Que modelos y que configs
%como lo usas en el sentido de tocar cosas, parametros como numero de arboles profundidad, el Alpha del lasso

\section{Results of the testing}
Results as graph/ tables - This would be to compare a model across various iterations. Model best-run comparisons, etc.
% a modo tabla unica comparando todas las iteraciones de cada technologia, y luego una conjunta con la mejor de todo (Si me hace falta mas, pues al apendice)

Error medio en todas las predicciones
Y percentiles (expectation shortfall tambien?)

\section{Low level discussion} % of the results
This would be the final explanation that I would give to a colleague of mine, with full details on how I have done everything



% Chapter 5 - Regulatory Framework (Marco Regulador)
\chapter{Regulatory Framework}
Not essential for this investigative project because we do not go to market.
\section{Data Availability}
Habria que hablar de las normas, pq si alguien lo usa es para ir a mercado

I need to talk about laws, what data is permissible to use and how. This would be essential for a project that does indeed go to market.

For us, its not that relevant.

\section{Software \& Licenses}
Visual Studio Code - open source general coding needs \cite{vscode}

Git - a distributed version control system \cite{git}

GitHub - Repository hosting platform \cite{github}

Python - Programming Language \cite{python}

venvs??? \cite{python_venv}

Scikit-learn - library for ml \cite{scikit-learn}

Pandas - data management \cite{pandas}

ta library - Technical Analysis library\cite{ta-lib}

Only light use of Seaborn - data visualization \cite{seaborn}

NumPy - number manipulation \cite{numpy}

LaTeX - open source language used to create the final project document \cite{latex}

Overleaf - closed source platform used to compile LaTeX code \cite{overleaf}

macOS / Windows - closed source platforms across where the project was developed \cite{macos} \& \cite{windows}



% Chapter 6 - Socio-economic Environment - Entorno socio-economico
\chapter{Socio-economic Environment}
\section{Socio-economic Impact}
Why is this relevant?

- Energy Prediction
- Shutdowns
- Price sensitivity

\section{Project plan}
Gantt diagram representing time spent - Design and innovation time frames. Amortizar nuestras horas de trabajo mias y Emilio, ordenador y poco mas

\section{Budget}
Price breakdown of the investigation



% Chapter 7 - Conclusions
\chapter{Conclusions}
\section{Recap of the project}
After finishing the whole project, read it, and reintroduce the objectives to the reader. Remind them of the completion of them

\section{Revisit the objectives}
Predicting the prices in an accurate manner using Machine Learning algorithms and technical analysis

\section{Future work}
    Less error?

    Less 



%----------
%	Bibliography
%----------	

\clearpage
\addcontentsline{toc}{chapter}{Bibliography}

\printbibliography



%----------
%	Appendix
%----------	

% If your work includes Appendix, you can uncomment the following lines
%\chapter* {Appendix x}
%\pagenumbering{gobble} % Appendix pages are not numbered



\end{document}